\documentclass[a4paper,12pt]{article}

%%% Работа с русским языком
\usepackage{cmap}					% поиск в PDF
\usepackage{mathtext} 				% русские буквы в формулах
\usepackage[T2A]{fontenc}			% кодировка
\usepackage[utf8]{inputenc}			% кодировка исходного текста
\usepackage[english,russian]{babel}	% локализация и переносы
\usepackage{indentfirst}            % красная строка в первом абзаце
\frenchspacing                      % равные пробелы между словами и предложениями

%%% Дополнительная работа с математикой
\usepackage{amsmath,amsfonts,amssymb,amsthm,mathtools} % пакеты AMS
\usepackage{icomma}                                    % "Умная" запятая

%%% Свои символы и команды
\usepackage{centernot} % центрированное зачеркивание символа
\usepackage{stmaryrd}  % некоторые спецсимволы

\renewcommand{\epsilon}{\ensuremath{\varepsilon}}
\renewcommand{\phi}{\ensuremath{\varphi}}
\renewcommand{\kappa}{\ensuremath{\varkappa}}
\renewcommand{\le}{\ensuremath{\leqslant}}
\renewcommand{\leq}{\ensuremath{\leqslant}}
\renewcommand{\ge}{\ensuremath{\geqslant}}
\renewcommand{\geq}{\ensuremath{\geqslant}}
%\renewcommand{\emptyset}{\ensuremath{\varnothing}}


%\DeclareMathOperator{\sgn}{sgn}
%\DeclareMathOperator{\ke}{Ker}
%\DeclareMathOperator{\im}{Im}
%\DeclareMathOperator{\re}{Re}

\newcommand{\N}{\mathbb{N}}
\newcommand{\Z}{\mathbb{Z}}
\newcommand{\Q}{\mathbb{Q}}
\newcommand{\R}{\mathbb{R}}
\newcommand{\Cm}{\mathbb{C}}
\newcommand{\F}{\mathbb{F}}
\newcommand{\id}{\mathrm{id}}

%%% Колонтитулы


\usepackage{geometry}



\begin{document}

    \author{Олег}
    \title{Статья}

    %\maketitle
    \tableofcontents

    \section{Lecture 1}

    Доброе утро! \[ax = b\]

    Nice!
    \[\int_{-\infty}^{+\infty} e^{\frac{x^2}{2}} = \sqrt{2\pi} \]

    \[ \forall \exists \]

    \[ \frac54 \frac{stuff}2 \]

    \[()  \big( \big) \bigg( \bigg) \]
    \[\left( \right)\]

    \subsection*{Подраздел, но без номера}

    {\bfseries \itshapeЗдесь выделенное}, а здесь нет

    Модификаторы размера и шрифта:
    
    {\Large Большой текст}, текст поменьше, {\small вообще маленький текст жесть}

    {\rmfamily Шрифт}
    {\sffamily Шрифт}
    {\ttfamily Шрифт}
    {\scshape Шрифт}
    
    \subsection{Простейшие окружения}
    \begin{center}
    Этот текст будет центрирован.
    \end{center}
    \begin{flushright}
        А этот смещен вправо.
    \end{flushright}

    %https://www.overleaf.com/learn/latex/Theorems_and_proofs

    %"--- для разделения подлежащего и сказуемого
    --- для других ситуаций с тире в тексте
    -- для указания числовых промежутков
    - дефисы
    $-$ знак минус

    \section{Lecture 2}

    \subsection{Отступы}

    \hspace{5px} \vspace{5px} 
    \hspace*{5px} \vspace*{5px} –-- не сокращаемые концом строки
    \hfill \vfill – распорки

    \subsection{Переходы на новую строку}

    поговорим о жизни 
    \par или нет
    \\ или да
    \newline а все-таки

    \subsection[Секции]{Основные разделы}
    documentclass определяет разделы, которые можно использовать
    \begin{enumerate}
        \item part
        \item chapter –-- в article не рабоатет
        \item section
        \item subsection
        \item paragraph
        \item subparagraph
    \end{enumerate}

    % синтаксис: \section_name[short_title]{full_title}

    \subsection{Генерация разделов}

    \noindent maketitle –-- author, date, title
    \\ tableofcontents
    \\ listoffigures
    \\ listoftables
    \\ appendix

    $$\mathrm{id}$$


    
\end{document}
